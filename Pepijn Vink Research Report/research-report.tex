% Options for packages loaded elsewhere
\PassOptionsToPackage{unicode}{hyperref}
\PassOptionsToPackage{hyphens}{url}
\PassOptionsToPackage{dvipsnames,svgnames,x11names}{xcolor}
%
\documentclass[
]{interact}

\usepackage{amsmath,amssymb}
\usepackage{iftex}
\ifPDFTeX
  \usepackage[T1]{fontenc}
  \usepackage[utf8]{inputenc}
  \usepackage{textcomp} % provide euro and other symbols
\else % if luatex or xetex
  \usepackage{unicode-math}
  \defaultfontfeatures{Scale=MatchLowercase}
  \defaultfontfeatures[\rmfamily]{Ligatures=TeX,Scale=1}
\fi
\usepackage{lmodern}
\ifPDFTeX\else  
    % xetex/luatex font selection
\fi
% Use upquote if available, for straight quotes in verbatim environments
\IfFileExists{upquote.sty}{\usepackage{upquote}}{}
\IfFileExists{microtype.sty}{% use microtype if available
  \usepackage[]{microtype}
  \UseMicrotypeSet[protrusion]{basicmath} % disable protrusion for tt fonts
}{}
\makeatletter
\@ifundefined{KOMAClassName}{% if non-KOMA class
  \IfFileExists{parskip.sty}{%
    \usepackage{parskip}
  }{% else
    \setlength{\parindent}{0pt}
    \setlength{\parskip}{6pt plus 2pt minus 1pt}}
}{% if KOMA class
  \KOMAoptions{parskip=half}}
\makeatother
\usepackage{xcolor}
\setlength{\emergencystretch}{3em} % prevent overfull lines
\setcounter{secnumdepth}{5}
% Make \paragraph and \subparagraph free-standing
\ifx\paragraph\undefined\else
  \let\oldparagraph\paragraph
  \renewcommand{\paragraph}[1]{\oldparagraph{#1}\mbox{}}
\fi
\ifx\subparagraph\undefined\else
  \let\oldsubparagraph\subparagraph
  \renewcommand{\subparagraph}[1]{\oldsubparagraph{#1}\mbox{}}
\fi


\providecommand{\tightlist}{%
  \setlength{\itemsep}{0pt}\setlength{\parskip}{0pt}}\usepackage{longtable,booktabs,array}
\usepackage{calc} % for calculating minipage widths
% Correct order of tables after \paragraph or \subparagraph
\usepackage{etoolbox}
\makeatletter
\patchcmd\longtable{\par}{\if@noskipsec\mbox{}\fi\par}{}{}
\makeatother
% Allow footnotes in longtable head/foot
\IfFileExists{footnotehyper.sty}{\usepackage{footnotehyper}}{\usepackage{footnote}}
\makesavenoteenv{longtable}
\usepackage{graphicx}
\makeatletter
\def\maxwidth{\ifdim\Gin@nat@width>\linewidth\linewidth\else\Gin@nat@width\fi}
\def\maxheight{\ifdim\Gin@nat@height>\textheight\textheight\else\Gin@nat@height\fi}
\makeatother
% Scale images if necessary, so that they will not overflow the page
% margins by default, and it is still possible to overwrite the defaults
% using explicit options in \includegraphics[width, height, ...]{}
\setkeys{Gin}{width=\maxwidth,height=\maxheight,keepaspectratio}
% Set default figure placement to htbp
\makeatletter
\def\fps@figure{htbp}
\makeatother
\newlength{\cslhangindent}
\setlength{\cslhangindent}{1.5em}
\newlength{\csllabelwidth}
\setlength{\csllabelwidth}{3em}
\newlength{\cslentryspacingunit} % times entry-spacing
\setlength{\cslentryspacingunit}{\parskip}
\newenvironment{CSLReferences}[2] % #1 hanging-ident, #2 entry spacing
 {% don't indent paragraphs
  \setlength{\parindent}{0pt}
  % turn on hanging indent if param 1 is 1
  \ifodd #1
  \let\oldpar\par
  \def\par{\hangindent=\cslhangindent\oldpar}
  \fi
  % set entry spacing
  \setlength{\parskip}{#2\cslentryspacingunit}
 }%
 {}
\usepackage{calc}
\newcommand{\CSLBlock}[1]{#1\hfill\break}
\newcommand{\CSLLeftMargin}[1]{\parbox[t]{\csllabelwidth}{#1}}
\newcommand{\CSLRightInline}[1]{\parbox[t]{\linewidth - \csllabelwidth}{#1}\break}
\newcommand{\CSLIndent}[1]{\hspace{\cslhangindent}#1}

\usepackage{orcidlink}
\makeatletter
\makeatother
\makeatletter
\makeatother
\makeatletter
\@ifpackageloaded{caption}{}{\usepackage{caption}}
\AtBeginDocument{%
\ifdefined\contentsname
  \renewcommand*\contentsname{Table of contents}
\else
  \newcommand\contentsname{Table of contents}
\fi
\ifdefined\listfigurename
  \renewcommand*\listfigurename{List of Figures}
\else
  \newcommand\listfigurename{List of Figures}
\fi
\ifdefined\listtablename
  \renewcommand*\listtablename{List of Tables}
\else
  \newcommand\listtablename{List of Tables}
\fi
\ifdefined\figurename
  \renewcommand*\figurename{Figure}
\else
  \newcommand\figurename{Figure}
\fi
\ifdefined\tablename
  \renewcommand*\tablename{Table}
\else
  \newcommand\tablename{Table}
\fi
}
\@ifpackageloaded{float}{}{\usepackage{float}}
\floatstyle{ruled}
\@ifundefined{c@chapter}{\newfloat{codelisting}{h}{lop}}{\newfloat{codelisting}{h}{lop}[chapter]}
\floatname{codelisting}{Listing}
\newcommand*\listoflistings{\listof{codelisting}{List of Listings}}
\makeatother
\makeatletter
\@ifpackageloaded{caption}{}{\usepackage{caption}}
\@ifpackageloaded{subcaption}{}{\usepackage{subcaption}}
\makeatother
\makeatletter
\@ifpackageloaded{tcolorbox}{}{\usepackage[skins,breakable]{tcolorbox}}
\makeatother
\makeatletter
\@ifundefined{shadecolor}{\definecolor{shadecolor}{rgb}{.97, .97, .97}}
\makeatother
\makeatletter
\makeatother
\makeatletter
\makeatother
\ifLuaTeX
  \usepackage{selnolig}  % disable illegal ligatures
\fi
\IfFileExists{bookmark.sty}{\usepackage{bookmark}}{\usepackage{hyperref}}
\IfFileExists{xurl.sty}{\usepackage{xurl}}{} % add URL line breaks if available
\urlstyle{same} % disable monospaced font for URLs
\hypersetup{
  pdftitle={Demo Taylor and Francis template},
  pdfauthor={Pepijn Vink},
  pdfkeywords={template, demo},
  colorlinks=true,
  linkcolor={blue},
  filecolor={Maroon},
  citecolor={Blue},
  urlcolor={Blue},
  pdfcreator={LaTeX via pandoc}}

\title{Demo Taylor and Francis template}
\author{Pepijn
Vink$\textsuperscript{1}$~\orcidlink{0000-0001-6960-9904}}

\thanks{CONTACT: Pepijn
Vink. Email: \href{mailto:p.a.vink@uu.nl}{\nolinkurl{p.a.vink@uu.nl}}. }
\begin{document}
\captionsetup{labelsep=space}
\maketitle
\textsuperscript{1} Methodology and Statistics for the Behavioral,
Biomedical, and Social Sciences, Utrecht University,  
\begin{abstract}
This document is only a demo explaining how to use the template.
\end{abstract}
\begin{keywords}
\def\sep{;\ }
template\sep 
demo
\end{keywords}
\ifdefined\Shaded\renewenvironment{Shaded}{\begin{tcolorbox}[sharp corners, interior hidden, borderline west={3pt}{0pt}{shadecolor}, boxrule=0pt, breakable, frame hidden, enhanced]}{\end{tcolorbox}}\fi

\hypertarget{introduction}{%
\section{Introduction}\label{introduction}}

\hypertarget{methods}{%
\section{Methods}\label{methods}}

\hypertarget{causal-models}{%
\subsection{Causal Models}\label{causal-models}}

For all simulations, the following data generating mechanism will be
simulated:

\begin{equation}\protect\hypertarget{eq-matrix-t}{}{
\textbf{X}_t = \boldsymbol{\Phi}\textbf{X}_{t-1} + \textbf{B}_{ct}\textbf{C}^T + \textbf{1}\textbf{u}_t + \textbf{E}_t
}\label{eq-matrix-t}\end{equation}

\[
\textbf{X}_1 = \textbf{B}_{c1}\textbf{C}^T + \textbf{1}\textbf{u}_1 + \textbf{E}_1
\]

\begin{equation}\protect\hypertarget{eq-formula-t}{}{
\begin{bmatrix}
x_{it}\\
y_{it}
\end{bmatrix}
=
\begin{bmatrix}
\phi_{xx} & \phi_{xy}\\
\phi_{yx} & \phi_{yy}
\end{bmatrix}
\begin{bmatrix}
x_{i,t-1}\\
y_{i,t-1}
\end{bmatrix}
+
\begin{bmatrix}
\beta_{xc_1t} & \beta_{xc_2t}\\
\beta_{yc_1t} & \beta_{yc_2t}
\end{bmatrix}
\begin{bmatrix}
C_{1i}\\
C_{2i}
\end{bmatrix} +
U_{it} +
\begin{bmatrix}
\epsilon_{xit}\\
\epsilon_{yit}
\end{bmatrix}
}\label{eq-formula-t}\end{equation}

For \(t = 2, ..., T\) and

\[
\begin{bmatrix}
x_{i1}\\
y_{i1}
\end{bmatrix}
=
\begin{bmatrix}
\beta_{xc_11} & \beta_{xc_21}\\
\beta_{yc_11} & \beta_{yc_21}
\end{bmatrix}
\begin{bmatrix}
C_{1i}\\
C_{2i}
\end{bmatrix} +
U_{i1} +
\begin{bmatrix}
\epsilon_{xi1}\\
\epsilon_{yi1}
\end{bmatrix}
\]

\[
\begin{bmatrix}
\epsilon_{xit}\\
\epsilon_{yit}
\end{bmatrix}
\sim
\mathcal{N} \left(\begin{bmatrix} 0\\ 0 \end{bmatrix}, \begin{bmatrix} \psi_x & 0\\ 0 & \psi_y \end{bmatrix} \right)
\]

\[
\begin{bmatrix}
C_{1i}\\
C_{2i}
\end{bmatrix} \sim \mathcal{N}\left(\begin{bmatrix} 0\\0 \end{bmatrix}, \begin{bmatrix}\psi_{C_1} & 0 \\0 & \psi_{C_2} \end{bmatrix} \right)
\]

\[
U_{it} \sim \mathcal{N}\left(0, \psi_u\right)
\]

For \(i = 1,...,N\).

This SCM is similar to the DPM as it does not have an explicit
decomposition of within and between effects and it is thus an
observation based model, rather than a residual based model. One way
that it differs from the DPM, other than having an observed confounder
instead of a latent factor, is that the DPM includes covariances between
the residuals at each timepoint. Because our model is specified as a
DAG, two-headed arrows are not included, and this is thus expressed as a
latent confounder between the residuals at each timepoint. It can be
shown that when these the effect of these confounders on the residuals
are 1, this specification using a confounder \(u_t\) with
\(\sigma_{u_t}^2 = \psi_u\) is equivalent to specifying a covariance
between the residuals where \(\sigma_{x_ty_t} = \psi_u\). Furthermore,
the variance of the residuals at each timepoint, \(\psi_x\) and
\(\psi_y\) are equal to the residual variances in the DPM, minus the
variance of the unique factor \(u_t\) at that timepoint. The SCMs that
will be simulated are described below.

\hypertarget{one-confounder-time-invariant-effects.-model-1}{%
\subsubsection{One confounder, time-invariant effects. (Model
1)}\label{one-confounder-time-invariant-effects.-model-1}}

When there is only one confounder C, Equation~\ref{eq-matrix-t} reduces
to:

\[
\textbf{X}_t = \boldsymbol{\Phi}\textbf{X}_{t-1} + \beta_{ct}\textbf{c}^T + \textbf{1}\textbf{u}_t + \textbf{E}_t
\]

Equation~\ref{eq-formula-t} reduces to:

\[
\begin{bmatrix}
x_{it}\\
y_{it}
\end{bmatrix}
=
\begin{bmatrix}
\phi_{xx} & \phi_{xy}\\
\phi_{yx} & \phi_{yy}
\end{bmatrix}
\begin{bmatrix}
x_{i,t-1}\\
y_{i,t-1}
\end{bmatrix}
+
\begin{bmatrix}
\beta_{xct}\\
\beta_{yct}
\end{bmatrix}
C_{i} +
U_{it} +
\begin{bmatrix}
\epsilon_{xit}\\
\epsilon_{yit}
\end{bmatrix}
\]

The following choices were made for the parameters for the simulation:
\(\boldsymbol{\Phi} = \begin{bmatrix} \phi_{xx} & \phi_{xy}\\ \phi_{yx} & \phi_{yy} \end{bmatrix} = \begin{bmatrix} 0.1 & 0.15\\ 0.1 & 0.6 \end{bmatrix}\),
\(\beta_{ct} = \begin{bmatrix} 0.10\\0.12 \end{bmatrix}\),
\(\psi_u = 0.6\), \(\psi_x = \psi_y = 1 - \psi_u = 0.4\), and
\(psi_C = 5\).

\hypertarget{results}{%
\section{Results}\label{results}}

\hypertarget{discussion}{%
\section{Discussion}\label{discussion}}

\newpage{}

\hypertarget{references}{%
\section*{References}\label{references}}
\addcontentsline{toc}{section}{References}

\hypertarget{refs}{}
\begin{CSLReferences}{0}{0}
\end{CSLReferences}



\end{document}
